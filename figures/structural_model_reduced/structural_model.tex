\documentclass[tikz, convert=false, border=10pt]{standalone}

\usepackage[utf8]{inputenx}%  http://ctan.org/pkg/inputenx
% Euler for math | Palatino for rm | Helvetica for ss | Courier for tt
% \renewcommand{\rmdefault}{ppl}% rm
% \linespread{1.05}% Palatino needs more leading
\usepackage[scaled]{helvet}% ss //  http://ctan.org/pkg/helvet
% \usepackage{courier}% tt // http://ctan.org/pkg/courier
% \usepackage{eulervm}  %  http://ctan.org/pkg/eulervm
% a better implementation of the euler package (not in gwTeX)
% \normalfont%
\usepackage[T1]{fontenc}%  http://ctan.org/pkg/fontenc
% \usepackage{textcomp}%  http://ctan.org/pkg/textcomp

\usetikzlibrary{shapes, arrows.meta, positioning, calc, bending}
\usepackage{circuitikz}

\def\Plus{\texttt{+}}
\def\Minus{\texttt{-}}

\begin{document}
\tikzset{block/.style = {draw, rectangle, thick, inner sep=0pt,
                         minimum height = 3em, minimum width = 3em},
         input/.style = {coordinate},
         output/.style = {coordinate},
         sum/.style={draw, thick, circle, path picture={%
         % https://tex.stackexchange.com/questions/47263/tikz-node-with-multiple-shapes
         \draw[thick, black]
            (path picture bounding box.north west) -- (path picture bounding box.south east)
            (path picture bounding box.south west) -- (path picture bounding box.north east);
         }},
         line/.style ={draw, very thick,-},
}

\begin{tikzpicture}[auto, node distance = 1.5em]
  \node[input, name = input] {};
  \node[sum, right = of input] (sum_1) {};
  \node[block, right = of sum_1] (K_e) {$K_e$};
  \node[block, right = of K_e] (delay) {$e^{-{\tau_{0}} s}$};
  \node[sum, right = of delay] (sum_4) {};
  \node[block, right = of sum_4] (Y_NM) {$Y_{NM}$};
  \node[block, right = of Y_NM] (Y_FS) {$Y_{FS}$};
  \node[coordinate, right = 1.5em of Y_FS] (switch_three_right_connector) {};
  \node[block, right = 3em of Y_FS] (Y_c) {$Y_{c}$};
  \node[output, right =3em of Y_c] (M) {};
  \node[block, below = of Y_NM] (Y_PF) {$Y_{PF}$};
  \coordinate[right = 1em of Y_c, name = outer_loop_empty] {};
  \coordinate[below = 8em of outer_loop_empty, name = outer_loop_empty_lower] {};
  \draw[thick] (outer_loop_empty) -| (outer_loop_empty_lower);

  \node[] at (-0.5em,0) {$C$};
  % \node[] at (3.25em,0.75em) {$E$};
  % \node[] at (12.5em,0.75em) {$E_c$};
  % \node[] at (14.75em,0.75em) {$E_m$};
  % \node[] at (19.5em,0.75em) {$\delta_F$};
  \node[] at (25em,0.75em) {$\delta_M$};
  \node[] at (33em,0em) {$M$};
  % \node[] at (27em,-2.5em) {$U_M$};

  \node[text width=3em] at (22em,-6em) {\itshape{proprioceptive\\feedback}};
  \node[] at (8em,-7.25em) {\itshape{visual feedback}};

  \draw[thick, dashed] (3em, 2.5em) -- (3em, -9em);
  \draw[thick, dashed] (3em, -9em) -- (28em, -9em);
  \draw[thick, dashed] (28em, -9em) -- (28em, -2.5em);
  \draw[thick, dashed] (28em, -2.5em) -- (19.5em, -2.5em);
  \draw[thick, dashed] (19.5em, -2.5em) -- (19.5em, 2.5em);
  \draw[thick, dashed] (19.5em, 2.5em) -- (3em, 2.5em);

  \node[] at (10em, 3.25em) {\itshape{PILOT}};
  \node[] at ($(Y_c)+(0, 2.25em)$) {\itshape{VEHICLE}};

  \node[] at (1.1em,.5em) {\Plus};
  \node[] at (1.5em,-.9em) {\Minus};

  \node[] at (12.8em,.5em) {\Plus};
  \node[] at (13.1em,-.9em) {\Plus};

  \draw[thick, -latex] (K_e) -- (delay);
  \draw[thick, -latex] (switch_three_right_connector) |- (Y_PF);
  \draw[thick, -latex] (Y_FS) -- (Y_c);
  \draw[thick, -latex] (Y_PF) -| (sum_4);
  \draw[thick, -latex] (input) -- node[] {} (sum_1);
  \draw[thick, -latex] (sum_1) -- (K_e);
  \draw[thick, -latex] (delay) -- (sum_4);
  \draw[thick, -latex] (sum_4) -- (Y_NM);
  \draw[thick, -latex] (Y_NM) -- (Y_FS);
  \draw[thick, -latex] (Y_c) -- (M);
  \draw[thick, -latex] (outer_loop_empty_lower) -| (sum_1);

\end{tikzpicture}
\end{document}
