\chapter{Feedback for Training Flight Tasks}
\label{chapter:aircraftfeedback}
% John A. Karasinski, Stephen K. Robinson
% Department of Mechanical and Aerospace Engineering, University of California, Davis

The literature has shown that a variety of tasks can benefit from concurrent feedback, and that more functionally complex tasks tend to see a larger performance improvement.
This result stems from results of many different experiments and researchers, however, and it is uncommon for experiments to explore the interaction effect between feedback and functional task complexity directly.
We designed this experiment to directly address this gap in the literature by evaluating the effects of feedback on an aircraft flight task with one, two, or three degrees of freedom.
Portions of this chapter were originally published in the conference proceedings for the Human Factors and Ergonomics Society 2019~\citep{RN42}.

\section{Introduction}
Augmented feedback, information that relates an individual's performance to a desired performance, has been found to generally enhance motor learning in a wide variety of manual motor control tasks~\citep{salmoni_knowledge_1984}.
Many feedback modalities and implementations have been investigated in the literature, some of which have been found to be more effective than others.
One of the key aspects to successfully implementing feedback is knowing when to provide feedback to the participant.
Feedback can be provided concurrently, in real-time as the task is executed; or terminally, after the task is completed.
Visual concurrent feedback, for example, has been shown to greatly enhance motor learning as task complexity increases, while terminal feedback is better suited for tasks with low functional complexity~\citep{sigrist_augmented_2013}.
As \citeauthor{sigrist_augmented_2013} note in their review of augmented visual, auditory, haptic, and multimodal feedback, however, "[u]p to now, mostly low-dimensional, simple, and rather artificial labor tasks have been investigated even though, in real life, most motor tasks are multidimensional and complex."

Aircraft and spacecraft flight-control tasks are complex, multidimensional challenges for human manual control, and present both demanding learning requirements and high cognitive demands.
In the pursuit of improving pilot performance during training, several researchers have investigated the effects of adding visual and/or audio feedback to flight displays.
\citeauthor{doi:10.1518/001872096778940859} explored the use of auditory displays in a 3D flight task.
By adding auditory displays to the existing simulation, they were able to reduce search time in an aircraft location and tracking task.
Similarly, \citeauthor{doi:10.1207/s15327108ijap1403} showed that U.S. Air Force pilots could better maintain flight parameters and report reduced workload with the use of multisensory cueing.
Of the many studied feedback strategies, concurrent bandwidth feedback is among the most promising for complex tasks.
Concurrent bandwidth feedback is presented in real-time, during task execution, but only when some variable deviates outside of a defined bandwidth of acceptable values.
Researchers have used concurrent bandwidth feedback to study participants ability to learn to drive a vehicle, having found it to be effective at improving lane keeping~\citep{de_groot_effect_2011}.

At UC Davis, our recent experiments with concurrent bandwidth feedback in complex manual tasks have resulted in large improvements in human performance with an added benefit of reduced workload.
Our experiment with simulated spacecraft-piloting investigated the effects of concurrent bandwidth feedback on a complex, four-degree-of-freedom manually controlled on-orbit inspection task~\citep{karasinski_real-time_2017}.
We found that simple visual feedback on the controlled degrees of freedom improved initial and fully trained performance while reducing inferred and self-reported workload.
In fact, participants in the feedback group performed as well in their first trial as participants in the control group did after two hours of training.
Our recent work also includes an investigation into the effectiveness of concurrent bandwidth feedback for learning a 3D joystick-controlled tracking task in augmented reality~\citep{karasinski_evaluating_2019}.
This experiment investigated whether concurrent bandwidth feedback could teach participants to interpret 3D depth cues.
Our results suggested that participants who were exposed to visual concurrent bandwidth feedback early on sustained improved performance through the duration of the experiment compared to participants that began in a baseline condition without feedback.
Through these previous experiments, we have shown that concurrent bandwidth feedback can be effective at improving human performance for complex manual tasks.
The very complex spacecraft piloting experiment also showed large reductions in cognitive workload, though this effect was not observed in the 3D tracking experiment, which had much lower functional task complexity.
In the research reported here, we build upon our previous work and that in the literature by investigating an operationally relevant, joystick-commanded flight-control task with the objective of determining the effect of task complexity on the influence of concurrent bandwidth feedback.

In our current experiment, subjects controlled a simulated aircraft with realistic flight dynamics through a series of tasks of increasing functional complexity.
By allowing for multiple levels of functional complexity, we can investigate what level of complexity is required to observe changes in human performance and cognitive workload.
This experiment also investigated the effects of removing concurrent feedback after training to evaluate changes in performance and workload during participants' immediate retention.
The retention portion of the experiment was performed to investigate the guidance hypothesis, which states that consistent feedback during the acquisition phase of learning leads to a dependency on the feedback~\citep{salmoni_knowledge_1984}.

\section{Method}
\subsection{Task}
\subsubsection{Control Modes}
Participants were tasked with flying a simulated Boeing 747 aircraft in three control modes. In order of increasing degrees of freedom and functional complexity, these three control modes were:
\begin{itemize}
    \item[\textbf{P}] Pitch only (low complexity)
    \item[\textbf{PR}] Pitch and Roll (moderate complexity)
    \item[\textbf{PRA}] Pitch, Roll and Altitude (significant complexity)
\end{itemize}
Depending on the control mode, participants were required to use a joystick to null disturbances in pitch, roll, and/or to maintain a constant altitude.
Participants were informed that all three tasks were equally important, and to try not to neglect or prioritize individual tasks.

Each participant completed a total of 36 trials; 12 in each of the three control modes.
Each trial had a duration of 82 seconds, and participants self-initiated the trial by activating a trigger on the joystick.
The trial order was designed such that each participant flew the simulator in increasing order of task complexity, with the sequence of P, PR, PRA, P, PR, $\ldots$, PRA.
This design was chosen to provide exposure to each control mode as quickly as possible, such that we could capture the early learning phases of each mode.

\subsubsection{Forcing Functions}
Both the pitch and roll axes were affected by disturbance signals, resulting in a disturbance-rejection task.
The disturbance signal took the form of a quasi-random sum of sines (based on~\citet{doi:10.2514/1.39953}).
The same forcing function was used for disturbing both pitch and roll, though the roll disturbance function was temporally shifted by 85 seconds to minimize the correlation between resulting pitch and roll disturbances.
Aircraft altitude varied as a result of pitch variation.
The same disturbing function was used for every trial, though participants were naïve to this.

\subsubsection{Secondary Task}
To estimate participant workload more objectively than with a questionnaire, we added a secondary task that assesses the subject's cognitive margin available for attending to non-primary task execution.
The secondary task was displayed to the right of the flight-guidance display (see Figure~\ref{figure-hfes:userinterface}) and consisted of a teal colored indicator which changed color to blue or green at pseudorandom times.
Ten 8-second secondary-task windows were displayed during each 82-second long trial.
The indicator would randomly change during this interval, providing participants up to 5 seconds to respond.
The pseudorandom times and colors for the secondary task were identical for each participant.
This secondary task has been validated in previous studies, which have shown it to correlate well with participants' subjective workload estimates~\citep{hainley_pilot_2013}.

\subsection{Simulator}
\begin{figure}[b!]
    \begin{center}
        \includegraphics[width=0.8\linewidth]{figures/Aircraft/image1.png}
        \caption[The user interface]{The user interface consisting of the attitude indicator, altimeter, and secondary task.}
        \label{figure-hfes:userinterface}
    \end{center}
\end{figure}

\begin{figure}[b!]
    \begin{center}
        \includegraphics[width=0.8\linewidth]{figures/Aircraft/image2.png}
        \caption[A participant seated in front of the simulator display]{A participant seated in front of the simulator display, controlling the flight task with the joystick.}
        \label{figure-hfes:participant}
    \end{center}
\end{figure}

Participants were seated at a fixed-base simulator for the duration of the experiment, see Figure~\ref{figure-hfes:participant}.
The simulator consisted of a 30-inch monitor and a single joystick.
The user interface was centered in the display and presented to the user at 1000x500 pixels.
The user interface consists of a traditional aircraft attitude indicator on the left, an altimeter in the middle, and the secondary task on the right.
The Boeing 747 was modeled using data available in NASA CR-2144~\citep{heffley1972aircraft}.
Flight Condition 2 was chosen for this experiment, and represents slow, sea-level flight, with the landing gear retracted and with flaps extended to 20 degrees.
Longitudinal and lateral body axis derivatives were converted to the stability axis and implemented in a state-space model~\citep{stevens2015aircraft}.
The values from NASA CR-2144 for this flight mode were plugged into the longitudinal and lateral dynamics matrices presented in Appendix~\ref{appendix:dynamics}.
Participants used the elevators and ailerons to control pitch and roll, respectively.
Altitude was controlled by the subject with time-integrated pitch commands, as in a real aircraft.

\subsection{Experimental Design}
Participants were evenly split into two groups: a control group and a feedback group.
Participants in the feedback group received concurrent bandwidth feedback on the first 27 trials (9 in each control mode) and then flew the last 9 trials (3 in each control mode) without feedback to test their immediate retention.
Participants in the control group never received concurrent bandwidth feedback, nor was it mentioned to them during the study.

For participants experiencing feedback, feedback was presented based on the flight control mode of their current trial.
This feedback was implemented by changing the color of an indicator's elements from yellow to red when it deviated from outside of its allowed bandwidth and returning the indicator to yellow when it returned within the bandwidth.
Acceptable bandwidths of 3 degrees for pitch and roll, and 30 feet for altitude were chosen based on preliminary testing.
The choice of these bandwidths was based on preliminary pilot studies, and is further explored in Chapter~\ref{chapter:bandwidthstudy}.
Pitch feedback occurred on the center dot.
Roll feedback was shown on the wings and the roll indicator at the top of the attitude indicator.
Altitude feedback was enabled on the background color of the altimeter.
In Figure~\ref{figure-hfes:userinterface}, all three parameters are shown inside the acceptable bandwidth, and are therefore displayed in yellow.

Participants began the experiment by signing a consent form, then filled out a survey with demographic questions, then had a brief, pre-recorded training session.
After this training, participants immediately began the experiment and progressed through the 36 trials at their own pace.
Participants noted their workload on a piece of paper after each trial.
Subjects in the feedback group were paused after the 27th trial, at which time the proctor explained that the feedback would no longer appear, but that they ``should continue to perform the task to the best of [their] ability.'' Subjects in the feedback group also filled out a questionnaire after the end of the experiment trials which asked them about their experience with the feedback.

\subsubsection{Independent variables}
The three independent variables in this experiment were Group, Mode, and Trial.
Group, a between subjects factor, described if subjects received feedback --- Control or Feedback.
Mode, a within subjects factor, was the three different control modes --- P, PR, or PRA.
Trial, also a within subjects factor, was the trial that subjects repeated 12 times in each mode.

\subsubsection{Dependent measures}
The root-mean-square error (RMSE) of pitch was calculated for every trial.
This allowed for a consistent measurement across every trial as the same pitch disturbance was present regardless of the control mode.
The roll RMSE was calculated for each PR and PRA trial, and the altitude RMSE was calculated for each PRA trial.
The RMSE values provide an objective measurement of performance.
The secondary task was activated ten times per trial.
Participants had five seconds to correctly respond to the secondary task once it changed color.
We recorded the rates for correct and incorrect responses, as well as lack of response.
We used the average correct response time as objective indication of workload.
The Modified Bedford Workload Scale is a ten-point subjective workload measurement tool~\citep{roscoe_subjective_1990}.
Participants were asked to follow the scale and record their workload after each trial, allowing us to observe changes in workload during training.

\subsection{Hypotheses}
We had three major hypotheses for this experiment:
\begin{itemize}
    \item[\textbf{H1.}] Participants in the feedback group will immediately outperform those in the control group. We expect this effect to be most pronounced for the most complex mode and to see little to no improvement for simple trials.
    \item[\textbf{H2.}] Participants in the feedback group will have lower workload than participants in the control at the end of the experiment. We expect this effect to be most pronounced for the most complex mode and to see little to no improvement for simple trials.
    \item[\textbf{H3.}] Participants in the feedback group will not suffer from the guidance hypothesis and will retain their performance and workload levels when the feedback is removed in the immediate retention trials.
\end{itemize}
These hypotheses were established from the literature, our previous experiments with feedback, and early pilot studies with this simulation framework.

\section{Results}
\subsection{Participants}
Participants in the experiment were 30 engineering students from the University of California, Davis (23 men, M = 23.0 years, SD = 4.4; 7 women, M = 22.6 years, SD = 3.0).
All participants had normal or corrected-to-normal vision and full motor control of their upper bodies.
Eighty percent of participants had previously used a joystick, 43\% had spent time in flight simulators, and 30\% had prior flight experience.
Both gender and participants with flight experience were counterbalanced between the two groups.
Written informed consent was obtained prior to testing in accordance with the University of California, Davis Institutional Review Board (Project \#1399789-1).

\subsection{Analysis}
Mixed models were used to calculate the significance of factors in our analysis due to the presence of performance outliers which were removed from the analysis.
The Satterthwaite method was used to calculate the adjusted degrees of freedom using the lmerTest package in R~\citep{RN53}.
When significant effects were observed, post hoc comparisons using the Tukey Honest Significance Difference (HSD) test were performed and considered significant at the p $<$ .05 level, and the Satterthwaite method was again used to calculate the degrees of freedom.
Only 7 of the 1080 total trials (30 subjects with 36 trials per subject) were removed.
These trials were extreme performance outliers, and including these trials does not change the primary results of the study.
A three-factor (Group, Mode, and Trial) mixed model with two repeated measures (Mode and Trial) was run on the pitch root-mean-square error.
There were significant main factors of group ($F(1, 27.97) = 6.3, p = 0.018$), mode ($F(2, 53.47) = 29.7, p < .001$), and trial ($F(11, 300.29) = 48.4, p < .001$).
There were also significant interaction effects between group and trial ($F(11, 300.29) = 2.5, p < 0.01$) and between mode and trial ($F(22, 601.58) = 2.8, p < .001$).
Despite the presence of interaction effects that result from participants learning the task (as indicated by the trial factor), the main effects can still be interpreted, see Figure~\ref{figure-hfes:pitchrmse}.
A Tukey test showed that the participants in the groups differed significantly, with the participants in the feedback group outperforming those in the control group ($M = 2.35, 3.05$, respectively, $SE = 0.20$).
An additional Tukey test showed that the participants' performance in the modes differed significantly, and participants performed best in P mode, followed by the PR mode, and finally the PRA mode ($M = 2.30, 2.67, 3.14$ respectively, $SE = 0.15$).

This same analysis was completed on the roll root-mean-square error, with similar results.
There were significant main factors of group $(F(1, 28.00) = 8.8, p < 0.01)$, mode $(F(1, 27.93) = 6.8, p = 0.015)$, and trial $(F(11, 308.22) = 19.6, p < .001)$.
There was also a significant interaction effect between mode and trial $(F(11, 308.06) = 4.6, p < .001)$, see Figure~\ref{figure-hfes:rollrmse}.
Tukey tests showed that the participants' performance between the groups and the modes each differed significantly, with the participants in the feedback group again outperforming those in the control group ($M = 1.96, 2.43$, respectively, $SE = 0.11$), and performance was best in the PR mode followed by the PRA mode ($M = 2.15, 2.24$, respectively, $SE = 0.08$).
A two-factor (Group and Trial) mixed model with one repeated measure (Trial) was run on the altitude root-mean-square error.
There were significant main factors of group ($F(1, 27.54) = 5.2, p = 0.030$) and trial ($F(11, 301.57) = 11.4, p < .001$).
Tukey tests showed that the participants' performance between the groups differed significantly, with the participants in the feedback group again outperforming those in the control group ($M = 28.3, 50.2$, respectively, $SE = 6.8$), and the trial effect showing learning throughout the experiment for both groups, see Figure~\ref{figure-hfes:altitudermse}.
See Figure~\ref{figure-hfes:completermse} for a plot of the root-mean-square error for each flight task in each mode.
A three-factor (Group, Mode, and Trial) mixed model with two repeated measures (Mode and Trial) was run on the modified Bedford workload scores.
There were significant main factors of mode ($F(2, 56) = 134.8, p < .001$), and trial ($F(11, 308) = 8.7, p < .001$), and a significant interaction effect between mode and trial ($F(22, 616) = 2.0, p < 0.01$).
Tukey tests showed that the participants' workload between the modes differed significantly, with workload lowest in P, then PR, and finally PRA ($M = 3.80, 4.99, 6.16$, respectively, $SE = 0.28$), and the trial effect representing slightly reduced workload throughout the experiment.

\begin{figure}[b!]
    \begin{center}
        \includegraphics[width=0.8\linewidth]{figures/Aircraft/image3.png}
        \caption[The mean Pitch RMSE for each trial]{The mean Pitch RMSE for each trial for participants in the P control mode. Data points are the mean, and error bars are the standard error of the mean.}
        \label{figure-hfes:pitchrmse}
    \end{center}
\end{figure}
\begin{figure}[b!]
    \begin{center}
        \includegraphics[width=0.8\linewidth]{figures/Aircraft/image4.png}
        \caption[The mean Roll RMSE for each trial]{The mean Roll RMSE for each trial for participants in the PR control mode. Data points are the mean, and error bars are the standard error of the mean.}
        \label{figure-hfes:rollrmse}
    \end{center}
\end{figure}
\begin{figure}[b!]
    \begin{center}
        \includegraphics[width=0.8\linewidth]{figures/Aircraft/image5.png}
        \caption[The mean Altitude RMSE for each trial]{The mean Altitude RMSE for each trial for participants in the PRA control mode. Data points are the mean, and error bars are the standard error of the mean.}
        \label{figure-hfes:altitudermse}
    \end{center}
\end{figure}

\begin{figure}[b!]
    \begin{center}
        \includegraphics[width=\linewidth]{figures/Aircraft/performance_measures.pdf}
        \caption[The root-mean-square error for each flight task in each mode]{The root-mean-square error for each flight task in each mode. Data points are the mean, and error bars are the standard error of the mean.}
        \label{figure-hfes:completermse}
    \end{center}
\end{figure}

\section{Discussion}

\begin{table}[tb]
    \centering
    \includetable{aircraft-perf-improvement.tex}
    \caption[Performance improvement of the feedback group over the control group]{Performance improvement of the feedback group over the control group at the end of the experiment for each flight RMSE metric.}
    \label{aircraft:perf-improvement}
\end{table}

Our analysis showed that participants in the feedback group performed significantly better than the control group in every trial across every metric and flight control mode.
For the first time that participants completed P, PR, and PRA trials, the feedback group performed 17.5\%, 31.7\%, and 37.4\% better than the control group according to the pitch RMSE metric.
Participants in the feedback not only immediately performed better, they also reached their peak performance much faster and had a final performance level which was significantly better than the control group, see Table~\ref{aircraft:perf-improvement}.
The largest performance improvement was seen in controlling altitude, where the feedback group had a final performance level which was 44.2\% better than the control group, confirming H1.
No group-related workload differences were detected in either the Modified Bedford scores or in the secondary task reaction times.
This suggests that, for tasks of this complexity, concurrent bandwidth feedback may not reduce workload~\citep{karasinski_evaluating_2019}.
Thus, for our second hypothesis, H2---we find that concurrent bandwidth feedback does not reduce workload in our flight tasks, and that tasks with higher functional complexity may be required to observe these effects.
Our third hypothesis was that subjects would primarily use the concurrent bandwidth feedback as a learning tool but that they would not become dependent on the feedback to the point that they required it to complete the task.
Retention tests are commonly used in the augmented feedback literature to verify that participants are not dependent on the feedback techniques.
Our analysis indicates no significant performance changes across any of our performance metrics when the feedback was removed, confirming H3.
After the experimental trials, we asked participants in the feedback group to complete a survey designed to identify when the feedback was or was not useful and how participants thought their performance and workload changed in the retention phase of the experiment.
One hundred percent of participants thought that the feedback helped them perform the task, 80\% reported that it helped them regain focus when their mind wandered, 73\% reported that it helped them to learn a scan pattern, and 27\% reported that the feedback was motivating.Survey results suggested that participants found the feedback especially useful in the roll and altitude tasks, which was reflected in the objective performance metrics.

\section{Conclusions}
Participants took part in a study investigating the effect of concurrent bandwidth feedback on flight performance and workload in three flight control modes of increasing complexity.
The participants in the feedback group performed significantly better than those in the control group, generally performing better on their second trial than those in the control group could by the end of the experiment.
Subjective and objective workload metrics showed no change in participant workload between the groups.
Survey questions identified that most participants found the feedback helpful in training them to establish a scan pattern, helping them to learn the task much quicker than those without feedback.
Feedback was removed for immediate retention trials, and participants showed no changes in performance or workload, suggesting that participants did not suffer from the guidance hypothesis.

This experiment provided additional insight into many of the research questions posed in Section~\ref{sec:intro_questions}.
Additionally, it confirms the result found generally in the literature and summarized in Figure~\ref{figure:sigrist_review}, that concurrent visual feedback is more effective at improving performance as functional workload increases.
By evaluating the effects of feedback on an aircraft flight task with multiple degrees of freedom, we were able to see 17.8\% to 44.2\% increases in performance, which increased as the degrees of freedom increased.
The lack of performance degradation when the feedback is removed, and therefore the rejection of guidance hypothesis, further illustrates that the concurrent bandwidth feedback is truly acting as feedback, not additional guidance.
This promotes the concept of the ``instructor model'' discussed in Section~\ref{section:cbf}, where the feedback becomes increasingly rare as pilot performance increases with training.

\chapter{Feedback Bandwidth Study}
\label{chapter:bandwidthstudy}

As we mentioned in Section~\ref{section:cbf}, several authors have noted that ``[b]andwidth feedback has been shown to be effective; however, setting the error threshold is not trivial''~\citep{timmermans_technology-assisted_2009, RIBEIRO2011231, sigrist_augmented_2013}.
While using an operational limit of acceptable performance can make the choice of setting the error threshold simpler, this option is not always available.
Additionally, mission designers often want to know what the optimum pilot performance is for mission planning, and instructors are interested in what levels of performance are acceptable during training.
In this Chapter, we treat the bandwidth as a variable in order to evaluate its effect on resultant subject performance and workload.

\section{Introduction}

Our previous work has shown that concurrent bandwidth feedback can be effective, but our choice of bandwidth has been ad-hoc in nature and primarily based on preliminary pilot studies.
These pilot studies have shown that there may be a wide variety of acceptable bandwidths, but it is unclear if there is a single bandwidth that would lead to optimal performance or a minimum learning time.
It is clear, however, that adverse thresholds which require unrealistic performance will not improve performance and may lead to degraded performance when compared to a no feedback condition.
Bandwidths that are too loose or not demanding enough will result in feedback that is sparse and may result in performance that resembles a no feedback condition.

\section{Method}

We designed a follow-up experiment to the experiment presented in Chapter~\ref{chapter:aircraftfeedback} which investigates the changes in performance that result from exposure to different bandwidths.
This experiment uses the same simulator and flying task as the previous experiment with a few changes.
The display, forcing functions, and secondary task were all identical to the previous experiment.
In this experiment, however, subjects only completed trials in the pitch and roll mode.

\subsection{Experimental Design}

In this experiment, subjects were randomly assigned to one of four groups.
These consisted of a control group, which received no feedback, and three feedback groups which received feedback during training.
Subjects in the feedback groups received concurrent bandwidth feedback along both the pitch and roll axes at 2, 3, or 4 degrees.
As with the previous experiment, the feedback on each axis was activated independently when the individual axis crossed over the threshold bandwidth.
Unlike the previous experiment, which only had a training and immediate retention phases, this experiment has four phases spread across two experiment sessions.
During the first session subjects completed 16 training trials followed by 8 immediate retention trials, during which time subjects in feedback conditions had their feedback removed.
Subjects were dismissed after completing the immediate retention trials and asked to return approximately 24 hours later for the second experiment session.
In the second session, subjects completed 8 retention trials without feedback, then completed 8 transfer task trials without feedback.
The only change between the transfer task and the other trials is the system dynamics of the controlled vehicle and the magnitude of the disturbance force.
In the transfer task, subjects were responsible for flying a Navion aircraft instead of a Boeing 747.
The Navion was modeled using data available in NASA CR-96008~\citep{teper1969aircraft}.
Compared to the Boeing 747, the Navion is a much smaller and lighter aircraft, which requires a significantly different control strategy to effectively control.
The disturbance force consisted of the same functions as in the rest of the experiment but was scaled down such that maximum pitch deflection under an uncontrolled scenario was approximately the same as with the Boeing 747 trials.

\subsubsection{Independent variables}
The two independent variables in this experiment were Group and Trial.
Group, a between subjects factor, described if and when subjects received feedback.
Trial, a a within subjects factor, was the trial that subjects repeated over the course of the experiment.

\subsubsection{Dependent measures}
The root-mean-square error (RMSE) of pitch and roll was calculated for every trial and provided an objective measurement of performance.
As in the previous experiment, the secondary task was activated ten times per trial and participants had five seconds to correctly respond to the secondary task once it changed color.
We used the average correct response time as objective indication of workload.
Participants were asked to follow the Modified Bedford scale and record their workload after each trial, allowing us to observe changes in workload during training.

\subsection{Hypotheses}
We had four hypotheses for this experiment:
\begin{itemize}
    \item[\textbf{H1.}] Participants in the feedback groups will immediately outperform those in the control group.
    \item[\textbf{H2.}] Participants will report the same workload between groups, which will gradually decrease with training.
    \item[\textbf{H3.}] Participants in the feedback group will not suffer from the guidance hypothesis and will retain their performance and workload levels when the feedback is removed in the immediate retention and 24 hour retention trials.
    \item[\textbf{H4.}] Participants will continue to perform at relatively similar levels when they complete the transfer task.
\end{itemize}

\section{Results}
\subsection{Participants}
In December 2019, coronavirus disease 2019, commonly referred to as COVID-19, began to spread and quickly resulted in a global pandemic.
COVID-19 is an infectious disease caused by severe acute respiratory syndrome coronavirus, is highly contagious, and is mainly spread during close contact.
As a result of this pandemic, the University of California, Davis was closed and the Institutional Review Board recommended that we act to ``limit transmission of the virus by delaying or otherwise modifying non-essential interactions,'' postponing subject testing.
We had anticipated testing approximately 10-15 subjects per group, for a total of 40-60 subjects.
The remainder of the results discussed here are preliminary and based on the 19 subjects whose data was collected before a shelter-in-place was ordered.

Participants in the experiment were 19 engineering students from the University of California, Davis (17 men, 1 woman, 1 decline to state) with an average age of 21.3 years (SD = 2.2).
All participants had normal or corrected-to-normal vision and full motor control of their upper bodies.
Participants with flight experience were counterbalanced between the two groups.
Of the 19 subjects, 18 returned the next day after an average of 24.93 hours (SD = 0.6).
This study was exempted by the University of California, Davis Institutional Review Board (Project \#1537932-1), and subjects were not compensated for their time.

\subsection{Analysis}

As in the previous study, linear mixed models were used to calculate the significance of factors in our analysis due to the presence of performance outliers.
These outliers were removed from the analysis and the Satterthwaite method was used to calculate the adjusted degrees of freedom using the lmerTest package in R.
When significant effects were observed, post hoc comparisons were performed using the Tukey Honest Significance Difference (HSD) test and considered significant at the p $<$ .05 level, and the Satterthwaite method was again used to calculate the degrees of freedom.
A two-factor (Group and Trial) mixed model with one repeated measure (Trial) was run on the pitch root-mean-square error.
There was a significant main factor of Trial ($F(31, 434) = 3.6, p < 0.001$) and Group was not significant ($F(3, 14) = 2.6, p = 0.09$).
Additionally, the interaction effect between Group and Trial was not significant ($F(93, 434) = 0.9, p = 0.78$).
This result indicates that subjects learned to perform the task over time but that the effect of Group was not significant given the effect size and current number of subjects.
Even with one half to one third of the anticipated number of subjects, however, the main effect of Group is already heading towards significance, as we would expect based on the results of the previous study.
Differences in groups are beginning to emerge when examining the plot, see Figure~\ref{figure-bw:pitchrmse}, though the error bars are too large to draw definite conclusions.

This analysis can also be performed on the other dependent measures, with similar results.
Most metrics closely track those seen in the previous study, though the low sample size makes it difficult to draw conclusions.
Due to their extremely preliminary nature, these results are not included here.

\begin{figure}[tb!]
    \begin{center}
        \includegraphics[width=\linewidth]{figures/Aircraft/Bandwidth-PitchRMSE.png}
        \caption[The mean Pitch RMSE for each trial]{The mean Pitch RMSE for each trial for participants. Data points are the mean, and error bars are the standard error of the mean.}
        \label{figure-bw:pitchrmse}
    \end{center}
\end{figure}

\section{Discussion}

Based on the preliminary findings presented in Figure~\ref{figure-bw:pitchrmse}, choosing an appropriate bandwidth can potentially have an effect on the efficacy of concurrent bandwidth feedback.
While we have yet to observe statistically significant effects between groups, current data suggests that having a looser bandwidth improves performance.
While it was expected that there may be an optimal choice of bandwidth to maximally reduce training times and improve performance, we hypothesized that this value would be lower than or approximately equal to three degrees, not a higher value, based off our pilot studies.

While we cannot make any definite claims based on this preliminary data and analysis, we can revisit data from the previous experiment to make further predictions.
Increasing the acceptable bandwidth of feedback results in the concurrent bandwidth feedback being presented less often, and the repetition in training results in subjects improving their performance and reducing their deviations during compensatory tracking.
We can use the disturbance force to estimate the percentage of time that subjects are exposed to the feedback as we presented subjects with the same disturbance force during each trial.
Figure~\ref{figure-aircraft:noinput_pitchfeedback} presents the percentage of time that pitch feedback would be active if subjects did not provide any control input whatsoever.
This suggests that at the beginning of training, subjects with a bandwidth of two, three, and four degrees would experience feedback 76\%, 62\%, and 47\% of the time, respectively, if they provided no input.

Subjects in the feedback group had their bandwidth set at three degrees in our previous experiment and were instructed to try and improve their performance whenever they saw that the feedback was activated.
By looking at the percentage of pitch errors that were above a given value, we can infer what percentage of time feedback would have been on when subjects had finished their training phase, regardless of chosen bandwidth.
Figure~\ref{figure-aircraft:pitchfeedback} presents the percentage of time that pitch feedback was active at the end of training, showing 12\% for the three-degree bandwidth subjects experienced.
It is unclear if subjects continue to improve until they reach a point where feedback appears approximately 12\% of the time, or if this is simply a limit of human motor control for this task.
In either case, the 4\% of time feedback would have been presented if a four-degree bandwidth was chosen is clearly within human control.
If the 12\% target is a true motivational limit, 4\% would likely inspire greater confidence in subjects, who may go on to report lower workload values.
The two-degree bandwidth, however, would only be attainable 29\% of the time by fully trained subjects in the previous experiment, and would likely have a demotivating or otherwise negative effect.

\begin{figure}[tb!]
    \begin{center}
        \includegraphics[width=\linewidth]{figures/Aircraft/no_input_feedback_on.pdf}
        \caption[The percentage of active pitch feedback with no input present]{The percentage of active pitch feedback with no input present. This represents the resulting aircraft motion due to the disturbance force.}
        \label{figure-aircraft:noinput_pitchfeedback}
    \end{center}
\end{figure}
\begin{figure}[tb!]
    \begin{center}
        \includegraphics[width=\linewidth]{figures/Aircraft/p34_feedback_on.pdf}
        \caption[The percentage of active pitch feedback time at the end of the previous study]{The percentage of active pitch feedback time at the end of the previous study, when the bandwidth was set to three degrees. The black line is the mean, and the shaded region is the standard error of the mean.}
        \label{figure-aircraft:pitchfeedback}
    \end{center}
\end{figure}
\begin{figure}[tb!]
    \begin{center}
        \includegraphics[width=\linewidth]{figures/Aircraft/Bandwidth-PitchFeedbackOn.png}
        \caption[The percentage of active pitch feedback time for each trial]{The percentage of active pitch feedback time for each trial for participants. Note that the Control group never received feedback. Data points are the mean, and error bars are the standard error of the mean.}
        \label{figure-bw:pitchfeedback}
    \end{center}
\end{figure}

Finally, we can take a preliminary look at how often subjects in this experiment experienced their feedback between the different bandwidth groups.
The percent of time that feedback was active at the beginning and end of training should provide some insight into how subjects respond to different bandwidth levels.
Figure~\ref{figure-bw:pitchfeedback} shows the percentage of active feedback time for each trial for subjects in the three feedback groups.
We can see how effectively subjects used the feedback on the first trial by looking at the difference between the value on this plot and that predicted by the no feedback case.
On the first trial, subjects in the three feedback groups tended to reduce their anticipated feedback exposure more the larger the bandwidth was.
Subjects in the four-degree bandwidth group, for instance, nearly halved their anticipated feedback exposure, as could be expected from their pitch RMSE presented in Figure~\ref{figure-bw:pitchrmse}.
Additionally, subjects in this group asymptote very close to the 4\% value predicted by subjects in the previous study.
The two- and three-degree bandwidth groups, however, continue to show much larger feedback exposure percentages.
The results for subjects in the three-degree bandwidth group are especially confusing, as they do not asymptote to the 12\% number shown by subjects in the previous experiment.
While the differences in experimental procedures could be the cause of this, we believe that the low sample size (5 subjects thus far in this experiment compared to 15 subjects in the previous experiment) are more likely the cause.

\section{Conclusions}
While the preliminary data collected for this experiment can be used to begin to explain the effects of using different bandwidths in concurrent bandwidth feedback, more subjects are needed to make definitive claims.
Briefly revisiting our hypotheses, we can start to predict the results of this study.
\begin{itemize}
    \item[\textbf{H1.}] While subjects in the three- and four-degree bandwidth groups will likely immediately outperform those in the control group, the two-degree bandwidth presents too much of a challenge for subjects to effectively use the concurrent bandwidth feedback.
          This indicates researchers should be careful to choose attainable bandwidths, or they may incorrectly conclude that concurrent bandwidth feedback, in general, does not work for their task.
    \item[\textbf{H2.}] Based on our early results, participants will report the same workload between groups, which will gradually decrease with training.
    \item[\textbf{H3.}] Participants in the feedback group will not suffer from the guidance hypothesis and will retain their performance and workload levels when the feedback is removed in the immediate retention and 24 hour retention trials.
    \item[\textbf{H4.}] It is still too early to make claims on the effects of concurrent bandwidth feedback on transfer of training.
\end{itemize}
Finally, we note that this study will resume testing when the University of California, Davis Institutional Review Board deems that testing can continue.
% This work will provide valuable insight into the effect of choosing an appropriate bandwidth for training motor control tasks.
