\chapter{Conclusion}
\label{chap:conclusion}

\section{Summary}

% \subsubsection{Trade Study}
Appropriate integration between automation and robotics systems and their human operators is essential for future space exploration.
The Human Factors and Behavioral Performance Element of NASA's Human Research Program requires a systematic understanding of the critical human-automation/robotic (HAR) integration, or HARI, design challenges for future space exploration.
In Chapter~\ref{chapter:tradestudy}: \nameref{chapter:tradestudy}, we present a trade study which reports the results of a systematic assessment of the spaceflight-relevant HARI technologies and research topics addressing critical gaps in spaceflight-relevant HARI knowledge, and prioritizes the research required for successful human performance and HAR integration.
We reviewed relevant literature across the past ten years and interviewed ten subject matter experts across industry and academia to investigate the current state of HARI technology, challenges facing development, the state of HARI research across a wide range of fields, and opportunities for advancing the state of the art through directed research.
This information was used to identify relevant HARI technologies and research topics, as well as factors to assess relative priority of HARI technologies.
We worked with NASA stakeholders to weight the factors relevant to assessing HARI specific technologies.
A multi-dimensional trade analysis was performed to objectively score HARI research topics and specific technologies to recommended investment priorities for NASA.

% \subsubsection{Testing Augmented Feedback in Augmented Reality}
Recent advances in computing hardware have enabled a new generation of mobile augmented reality devices which have the potential to improve human performance and reduce workload in a variety of tasks.
The aim of Chapter~\ref{chap:3dtracking}: \nameref{chap:3dtracking} was to investigate the effect of several factors on human performance and workload in a three-axis manual tracking task.
Twenty-four (24) engineering students at the University of California, Davis were randomly placed into a one of two device groups ($n=12$ per group): a 2D or 3D display.
Subjects in both groups evaluated three different displays in a random order: a baseline display, a concurrent bandwidth feedback display, and a rotated display.
Subjects' objective performance was evaluated using the root-mean-square error (RMSE) of the depth ($z$) axis.
Objective workload was measured by the response time to a two-choice task, and subjective workload was evaluated using the NASA Task Load Index (NASA-TLX).
Results of ANOVA analysis on the $z$ axis RMSE showed significant effects for design ($F(2, 36)=84.92, p<.001$), device ($F(1, 18)=7.22, p<0.015$), and start design ($F(2, 18)=4.81, p<0.021$).
The ANOVA also showed a significant interaction effect between design and starting design ($F(4, 36)=8.55, p<0.0001$), and a three way interaction between design, device, and starting design ($F(4, 36)=5.57, p<0.002$).
In general, there were no significant effects found for workload measurements.
Both concurrent bandwidth feedback and a rotated display resulted in superior performance when compared to a baseline display.
Providing a 3D display did not, in general, improve performance.
Subjects in the 3D display group and that had early exposure to the concurrent bandwidth feedback, however, were able to use the feedback to achieve superior performance.

% \subsubsection{Augmented Feedback in an Electromyography Task}
To further close the gap between human and automation, research has explored various forms of biophysical measurements for use as feedback control, such as electromyography (EMG).
It is critical to evaluate the effects of training methodologies on performance and the evolution of workload and trust in order to achieve a seamless human-automation system.
In Chapter~\ref{chap:emg}: \nameref{chap:emg} we developed a surface EMG control task to observe the effects of training methodology on the development of performance, workload, and trust.
In this study, 48 subjects learned to use a surface EMG command system to perform a Fitts's Law computer-based cursor-to-target task.
The training phase consisted of 120 Trials and the subsequent retention phase contained 40 Trials.
Subjects were divided into four groups: Control, Concurrent Feedback, Terminal Feedback, and Adaptive Threshold.
The Control group trained and learned through repetition using the visual feedback of the cursor position.
The Concurrent Feedback group received additional concurrent visual feedback during command input, whereas the Terminal Feedback group had supplementary visual feedback after command input.
The Adaptive Threshold group did not have any additional visual feedback, but experienced changes in the cursor control dynamics to induce motor learning adaptation.
There were significant differences in percent of successfully completed trials, workload, and trust during the training phase, particularly in the earlier portion.
The treatments were removed during the retention phase and there were no significant differences across groups.
The results indicated that 1) there was no significant effect from the guidance hypothesis (i.e. the subjects did not rely on the feedback to perform the task), 2) the relationship between the subject and system evolved differently depending on the group, and 3) all training methodologies achieved similar results by the end of the study.
We concluded that including concurrent feedback during manual task learning may provide both real-time performance improvement and an early learning plateau.

% \subsubsection{Augmented Feedback Benefits Scale with Functional Task Complexity}
In Chapter~\ref{chapter:aircraftfeedback}: \nameref{chapter:aircraftfeedback} the effects of concurrent bandwidth feedback on operator performance and workload was analyzed in training an aircraft flight control task.
In the experiment, participants completed a simulated flight task consisting of three complexity levels using traditional flight instruments.
Thirty participants were divided into equal sized control and feedback groups.
The control group controlled simulated aircraft motion with visual guidance for pitch, roll, and altitude provided by traditional flight instruments.
The feedback group received additional visual concurrent bandwidth feedback for each controlled degree of freedom.
For both groups, performance and workload measurements were evaluated to determine the effects of the feedback on subject learning rate and maximum skill level.
To assess short-term retention of learned skill for the feedback group, the concurrent feedback was removed, and performance was again evaluated.
Statistical analyses showed that participants in the feedback group immediately performed better than those in the control group, that the performance difference between the two groups was more pronounced for more complex tasks, and that final performance levels for the feedback group significantly exceeded that of the control group.
We found that concurrent bandwidth feedback does not reduce workload in our flight tasks, and that for the short periods tested, participants continued to perform at the same performance and workload levels when the feedback was removed.

% \subsubsection{Modeling the Effects of Concurrent Bandwidth Feedback}
Finally, in Chapter~\ref{chapter:modeling}: \nameref{chapter:modeling} the Structural Model of the human pilot was investigated to model the results of the aircraft feedback study to better understand how the subjects' performance was changed by exposure to the feedback.
We estimated the values of the parameters in the model using a novel parameter identification technique and used the model to estimate the crossover frequency of the combined pilot/vehicle system.
The results of this technique were validated using the crossover frequencies identified by an ARX technique, the comparison with which showed good agreement.
Linear mixed effect models were used to evaluate how each structural model parameter changed between groups over the course of the experiment.
The result of this statistical test led us to propose the extension of the model to include a new term, $K_f$, which accumulates with exposure to feedback, and which is linearly added to the normal error sensing and gain compensation, $K_e$.
We found evidence that this term accumulates when subjects are exposed to a critical amount of feedback over the course of a given trial, and that including this term allows for the difference in performance between the control and feedback groups.

\section{Research Questions}

In Chapter~\ref{sec:intro_questions}, we listed six research questions that we aimed to answer by this research.
Going into this research, our primary questions were
\begin{enumerate}
      \item Can concurrent bandwidth feedback (CBF) improve performance of complex manual control tasks?
            \begin{enumerate}
                  \item Can CBF reduce the required training time to peak performance?
                  \item Can CBF be removed after reaching peak performance without reducing subject performance (i.e., does the guidance hypothesis not hold)?
                        %   \item Does CBF improve performance in transfer of training tasks?
                  \item Can performance be increased without increasing workload?
            \end{enumerate}
      \item Can we develop a model of human performance which includes the effects of concurrent bandwidth feedback?
            \begin{enumerate}
                  \item Can we use this model to estimate operational limits?
            \end{enumerate}
\end{enumerate}
Here we summarize our answers to these questions.

\subsubsection{Can concurrent bandwidth feedback improve performance of complex manual control tasks?}
Yes.
The results presented in the three research studies in this dissertation, the augmented reality tracking task, surface electromyography task, and aircraft flight tested over one hundred subjects.
In each of these tasks, subjects exposed to concurrent bandwidth feedback had improved performance in their control tasks compared to the control groups.
The amount of performance improvement varied between the tasks but appears to increase with functional task complexity.
Performance improvement varied between 17.8\% and 44.2\% for the aircraft flight task, with the largest benefit appearing in the most complicated task mode (see Table~\ref{aircraft:perf-improvement}).
In the augmented reality tracking task, subjects that trained with the feedback while wearing the augmented reality headset were able to better perform the task when the feedback was removed, which was not the case for subjects that did not wear the headset.

\subsubsection{Can CBF reduce the required training time to peak performance?}
Yes.
Subjects exposed to the concurrent bandwidth feedback generally performed better on their second trial than subjects in the control group did by the end of the experiments.
This is especially clear in the percent of successful trials completed in the surface electromyography study.
In this experiment, subjects exposed to CBF achieved a nearly perfect score by the second block of trials, a feat that subjects in the control group never achieved (see Figure~\ref{figure:label3}).
This was also the case in the aircraft study, where subjects saw immediate performance improvements after the first trial (see Figures \ref{figure-hfes:pitchrmse}, \ref{figure-hfes:rollrmse}, and \ref{figure-hfes:altitudermse}).

\subsubsection{Can CBF be removed after reaching peak performance without reducing subject performance (i.e., does the guidance hypothesis not hold)?}
Yes.
Many forms of augmented feedback have been plagued by the ``guidance hypothesis''.
This term describes when subjects become reliant on the augmented feedback given during training such that they come to use the augmented feedback, rather than other indicators available in the task, such that their performance is reduced when the augmented feedback is removed.
The concurrent bandwidth feedback here does not seem to suffer from the guidance hypothesis.
This was hinted at early on in the augmented reality tracking task, where subjects that were initially exposed to the feedback continued to perform at the same level during their other conditions.
While there was a small, nonsignificant reduction in performance during the EMG task when the feedback was removed (see Figure~\ref{figure:label3}), we also saw a small, nonsignificant increase in performance when the feedback was removed in the aircraft task (see Figures \ref{figure-hfes:pitchrmse}).
These results indicate that visual concurrent bandwidth feedback does not suffer from the guidance hypothesis, and that this form of augmented feedback can be used to help train tasks of this type and difficulty.

The reason that this feedback works is likely that it naturally fades away as subjects perform the task better (see Figure~\ref{fig:feedback_kfd}, for example. which shows the average feedback exposure time as a function of trial).
Fading feedback has usually been considered effective but knowing how to set the fading rate has continued to be an unsolved issue.
While bandwidth feedback of this type only replaces that issue with another (how does one best set the bandwidth for a given task?), it does provide an easy way to directly tie the fading rate to subject performance.
Additionally, many tasks have known operational limits which can be used to set a bandwidth, which provides a simple way to justify bandwidth selection.
%
% \subsubsection{Does CBF improve performance in transfer of training tasks?}
% Work to answer this is in progress.

\subsubsection{Can performance be increased without increasing workload?}
Yes.
Subjective workload was measured using the NASA-TLX and/or Modified Bedford Workload Scores in all three of our experimental studies.
No significant results changes in workload were detected between the control and feedback conditions in any of our studies, though we were able to detect changes in workload as a function of training and functional complexity.
This suggests that the augmented feedback developed in this work is significant as it can simultaneously increase performance without increasing workload.
Concurrent bandwidth feedback likely doesn't increase workload as it is only present on elements of the display that are already available to the control group.
By pointing out elements of the display that subjects should be paying attention to at any given time, much in the way that an expert instructor might, we are simply redirecting attention to the most vital elements of the display rather than giving the subjects another task to attend to.
As we are not adding additional tasks for subjects to attend to, they can use the feedback to do more (perform better) without adding costs to their cognitive load.

\subsubsection{Can we develop a model of human performance which includes the effects of concurrent bandwidth feedback?}
Yes.
In Chapter~\ref{chapter:modeling} we were able to motivate and develop a model of human performance, based on Structural Model of the Human Pilot, which includes the effects of concurrent bandwidth feedback.
This model modifies the Structural Model to include one additional gain, $K_f$, which accumulates as subjects are exposed to the concurrent bandwidth feedback.
Evidence from the aircraft study was used to develop this model, which shows that the amount of time exposed to the feedback in each trial leads to the change in the $K_f$ term from trial to trial.
Using the modified Structural Model, differences between groups of subjects with and without feedback can be modeled as their performance changes through the training process.

\subsubsection{Can we use this model to estimate operational limits?}
Yes.
The primary effect of increasing $K_f$ is to increase the crossover frequency of the combined pilot/vehicle system.
One can use this model to explore different bandwidths and see the resulting effect it has on the crossover frequency.
As the upper limit of human performance is around 2 rad/s, any bandwidth which results in a higher value would result in unsustainable levels of workload.
Further research is required, however, to determine if $\dot{K_f}$ is task dependent, which is what is driving this change.

\section{Future Work}

This research focused on the effect of one type of augmented feedback, concurrent bandwidth feedback on human performance and workload in complex manual control tasks.
We primarily investigated how feedback works with manual tracking tasks, where we showed that performance increases were correlated with functional task complexity.
The largest performance gains occurred in the most complex tasks, where subjects used the feedback to improve dramatically over very short periods of time when compared to subjects in the control group.
An extension of this work could further investigate the utility of concurrent bandwidth feedback to improve performance in emergency or other off-nominal scenarios.
If operators could use concurrent bandwidth feedback in actual operation, this research may have the potential to reduce aviation accidents by increasing operator situational awareness or reducing workload inflight.
One limitation of this work, however, was that it focused only on inexperienced and untrained undergraduate students.
Future work should investigate if concurrent bandwidth feedback can improve performance in well-trained operators.
If feedback only works for naive subjects, then it may only be useful in the very early stages of training.

In this research, we showed that concurrent bandwidth feedback did not cause subjects to suffer from the guidance hypothesis.
While we investigated immediate retention, it is unclear how much performance benefit would be retained in longer term experiments.
We also provided a uniform number of trials with feedback for all subjects and did not attempt to schedule or remove the feedback pre-emptively for subjects that were already performing well.
Future work could investigate the minimum number of trials that subject need to be exposed to concurrent bandwidth feedback to retain a performance benefit.
We also did not investigate the effect of changing the acceptable performance bandwidth as subjects improved through training.
It is possible that future work could investigate the effect of progressively narrowing the bandwidth as subjects improve their performance.
This allow for further improvements in performance but may ultimately lead to impossible performance restraints if not scheduled carefully.

This research focused on feedback that was visually displayed to an operator while they were completing tasks.
Other researchers have investigated providing feedback to operators using other modalities, such as with audio, haptic, or multimodal feedback.
Future researchers could investigate what the effect of the modality of concurrent bandwidth feedback has on subject performance and could identify if multimodal feedback has greater advantages over unimodal feedback.
Multimodal feedback is generally thought to be more effective, but an experiment investigating the interaction between feedback modality and task complexity could show the optimal way to provide feedback.
Our previous experiment with the SAFER vehicle (4 degree of freedom task) showed workload improvements, while the pitch, roll, and altitude task presented in this dissertation (3 degree of freedom task) did not.
In addition to have one additional degree of freedom, the SAFER tasks also involved navigation.
The aircraft flight task only involved stabilizing the vehicle and holding an altitude, however, and making direct comparisons between the two is difficult.
Investigating tasks of increasing difficulty would also allow researchers to identify what level of task complexity is needed for subjects to report reduced levels of cognitive workload when exposed to feedback.

This work has spawned a large number of new questions worth investigating.
The questions brought up in this section include:
\begin{enumerate}
      \item Can the concurrent bandwidth feedback be turned on in emergency scenarios to assist operators in quickly recovering from off-nominal conditions?
      \item Does the concurrent bandwidth feedback improve performance in well-trained operators, or does this technique only work for naive or inexperience operators?
            %
      \item What is the minimum number of trials that subjects need to be exposed to concurrent bandwidth feedback?
      \item Is there an optimal way that the bandwidth can be scheduled as subjects improve through trials?
            %
      \item What is the effect of providing the concurrent bandwidth feedback through other modalities (audio, haptic, or a multimodal approach integrating these and/or visual)?
            %
      \item What level of functional complexity is needed to see reductions in workload for subjects exposed to the concurrent bandwidth feedback?
\end{enumerate}
