% Declarations for Front Matter
\title{Concurrent Bandwidth Feedback for Complex Manual Control Tasks}
\author{John A. Karasinski}

% Choices are September, December, March, June
\degreemonth{June}
\degreeyear{2020}

\committee{Stephen K. Robinson}{Ron A. Hess}{Zhaodan Kong}{}{}

%Your Graduate Group
\officialmajor{Mechanical and Aerospace Engineering}
\graduateprogram{Mechanical and Aerospace Engineering}

%%%%%%%%%%%%%%%%%%%%%%%%%%%%%%%%%%%%%%%%%%%%%%%%%%%%%%%%%%%%%%%%%%%%%%%%
\abstract{
  Augmented feedback has been shown effective in improving human subject performance for a variety of simple and low-dimensional tasks, but more complex and realistic tasks are rarely explored in the literature.
  Instead of simply adding additional guidance, augmented feedback alerts operators to critical features of a task that they may otherwise not be aware of.
  However, past research has resulted in a significant caution --- that many forms of augmented feedback lead to the guidance hypothesis, which manifests as decreased performance when the feedback is removed.\\
  \\
  The current research explores a novel approach to a specific type of augmented feedback, called concurrent bandwidth feedback, and how it might be effectively applied to avoid the guidance hypothesis for complex tasks.
  Concurrent bandwidth feedback is provided to an operator in real-time, during task execution, when a specific signal deviates outside of an acceptable range of values.
  This ``Instructor Model'' of feedback highlights display elements that need urgent attention from the operator, and consequently appears less frequently as operator performance improves.
  Through the analysis of four human-in-the-loop experiments, we show that subjects that are exposed to concurrent bandwidth feedback dramatically and immediately improve task performance.
  Through varying the functional task complexity in an aircraft flight task, we show that the improvements in operator performance increase with task complexity.
  By investigating immediate and 24-hour skill retention, we also show that concurrent bandwidth feedback does not result in the guidance hypothesis.
  While increased human performance usually results in higher levels of cognitive demand, subjective workload measurements and objective secondary task performance indicate that our subjects did not experience additional cognitive load when using our novel concurrent bandwidth feedback.\\
  \\
  Control theory-based modeling techniques were explored to further understand and predict the effects of concurrent bandwidth feedback.
  The Structural Model of the human pilot was extended to explain the increased task performance by introducing a new control block which models concurrent bandwidth feedback received by the operator.
  Using data from our aircraft flight task and the Structural Model, we show that exposure to the concurrent bandwidth feedback results in increased error sensing and gain compensation, raising the resultant crossover frequency and ultimately improving pilot performance.\\
  \\
  In all of our studies, subjects exposed to concurrent bandwidth feedback had higher levels of performance, did not experience increased workload, and did not suffer from the guidance hypothesis, indicating that concurrent bandwidth feedback can be used as an effective training technique for complex manual control tasks.
}

\acknowledgments{
  Thank you to my fellow Human/Robotics/Vehicle Integration and Performance Lab members, who elevated this work with their high standards.
  Thank you to Professor Robinson, who provided the inspiration for this project and has guided me throughout the whole process.
  Thank you to my committee members, Professor Hess and Professor Kong, who provided support in shaping the research.
  Thank you to Richard Joyce and Sarah O'Meara, who were always willing to discuss research over coffee.\\
  \\
  Thank you to the San Jose State University Research Foundation, who provided financial support.
  Thank you to the Link Foundation, who selected me for the Advanced Training and Simulation Fellowship and provided financial support.
  Thank you to NASA Ames Research Center's Human Systems Integration Divison for having me as a Pathways Intern, providing financial support, and the knowledge and expertise of your wonderful staff.\\
  \\
  Thank you to the subjects who volunteered their time and made this research possible.\\
  \\
  Thank you to my family, without whom I would, quite literally, not be here.
}
